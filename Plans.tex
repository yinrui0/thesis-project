%%%%%%%%%%%%%%%%%%%%%%%%%%%%%%%%%%%%%%%%%%%%%%%%%%%%%%%%%%%%%%%%%%%%%%%%%%%%%%%%
\documentclass[11pt]{article}

\usepackage[authoryear]{natbib}
\usepackage[top=2.5cm,bottom=2.5cm,left=2.5cm,right=2.5cm]{geometry}
\usepackage{color}
\usepackage{chemarr}
\usepackage{amssymb}
\usepackage{graphicx}
\usepackage{textcomp} 
\usepackage[gen]{eurosym}
\usepackage{amsmath}
\usepackage[margin=1.5cm]{caption}
\usepackage{amsmath,mathtools}
\usepackage{subcaption}
%\usepackage{ulem}
\usepackage[normalem]{ulem}
%\setlength{\belowcaptionskip}{-5pt}
%\setlength{\abovecaptionskip}{-8pt}
\usepackage{enumitem}

%\usepackage[autolinebreaks,useliterate]{mcode}
\usepackage[usenames, dvipsnames]{xcolor}
\colorlet{shadecolor}{gray!5}
\usepackage{listings}
\lstset{ 
	language=R,                     % the language of the code
	basicstyle=\scriptsize\ttfamily, 	  % the size of the fonts
	numbers=left,                   % where to put the line-numbers
	numberstyle=\tiny\color{Blue},  % the style that is used for the line-numbers
	stepnumber=1,                   % the step between two line-numbers.
	numbersep=5pt,                  % how far the line-numbers are from the code
	backgroundcolor=\color{shadecolor},  % choose the background color.
	showspaces=false,               % show spaces adding particular underscores
	showstringspaces=false,         % underline spaces within strings
	showtabs=false,                 % show tabs within strings adding particular underscores
	frame=tb,                   	  % adds a frame around the code
	rulecolor=\color{Black},        % if not set, the frame-color may be changed on line-breaks within not-black text (e.g. commens (green here))
	tabsize=2,                      % sets default tabsize to 2 spaces
	captionpos=b,                   % sets the caption-position to bottom
	breaklines=true,                % sets automatic line breaking
	breakatwhitespace=false,        % sets if automatic breaks should only happen at whitespace
	keywordstyle=\color{RoyalBlue}, % keyword style
	commentstyle=\color{OliveGreen},% comment style
	stringstyle=\color{ForestGreen} % string literal style
}

\newcommand{\nrtodo}[1]{{\color{blue} NR: #1}}

%%%%%%%%%%%%%%%%%%%%%%%%%%%%%%%%%%%%%%%%%%%%%%%%%%%%%%%%%%%%%%%%%%%%%%%%%%%%%%%%
\title{\textbf{Plan for this Semester}}
\author{Yinrui Li}
\date{}


%\maketitle


\begin{document}
	\maketitle
	%%%%%%%%%%%%%%%%%%%%%%%%%%%%%%%%%%%%%%%%%%%%%%%%%%%%%%%%%%%%%%%%%%%%%%%%%%%%%%%%
	
	\begin{enumerate}
		\item \textbf{Deadline of depositing thesis: April 28} 
		
		It is best to be sent to Tami by April 21 (1 week earlier).
		
		\item \textbf{Plans for Each Week}
		
		Jan 25 -- Jan 31: 
		
		Draw an outline and list things that can be put in the thesis;
		
		List related graphs;
		
		Start with the introduction (check papers related to BC aging timescales, cloud activities and climate effect.);
		
		\bigskip
		
		Feb 1 -- Feb 7:
		
		Attend the Santa Fe conference;
		
		Continue with the introduction; 
		
		\bigskip
		
		Feb 8 -- Feb 14:
		
		Finish the first draft of introduction;
		
		Start writing method (Model Specification);
		
		\bigskip
		
		Feb 15 -- Feb 28:
		
		Finish first draft of method (MAM4; condensation and coagulation scheme, mass transfer rate and aging timescales; PartMC parameterization);
		
		Start with Conclusions: BC burden and emissions.
		
		\bigskip
		
		Mar 1 -- Mar 7:		
		
		Conclusion: BC burden sensitivity to aging criterion (horizontal and vertical profiles for March and April);
		
		Conclusion: BC radiative forcing to the aging criterion;
		
		\bigskip
		
		Mar 8 -- Mar 14:
		
		Conclusion: BC aging timescales for surface and middle layers;
		
		Conclusion: sensitivity of linear regression (model vs parameterization) to the number of monolayers.
		
		\bigskip
		\bigskip
		\bigskip
		
		Mar 15 -- Mar 21:
		
		Conclusion: Comparing BC Mixing State from CAMChem to SP2 Measurements.
		
		\bigskip
		
		Mar 21 -- Mar 28:
		
		Finish the first draft of conclusion part.
		
		
		\bigskip
		
		Mar 29 -- April 5:
		
		Write abstract.
		
		
		\bigskip
		
		
		

	
	
	
	
	
	
	
	\end{enumerate}
	
	\clearpage
	
	\bibliographystyle{plain-local-srefid}
	\bibliography{refs}
	
	
	
	
	
	
	%%%%%%%%%%%%%%%%%%%%%%%%%%%%%%%%%%%%%%%%%%%%%%%%%%%%%%%%%%%%%%%%%%%%%%%%%%%%%%%%
	
	
	
\end{document}
%%%%%%%%%%%%%%%%%%%%%%%%%%%%%%%%%%%%%%%%%%%%%%%%%%%%%%%%%%%%%%%%%%%%%%%%%%%%%%%%
%%%%%%%%%%%%%%%%%%%%%%%%%%%%%%%%%%%%%%%%%%%%%%%%%%%%%%%%%%%%%%%%%%%%%%%%%%%%%%%%
